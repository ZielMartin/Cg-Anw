\newpage
	\section{\Large ZIELBESTIMMUNG}
	Das Benutzungsziel ist:
	\begin{itemize}
		\item
	\end{itemize}
	\subsection{Musskriterien}
	\begin{itemize}
		\item
	\end{itemize}
	\subsection{Abgrenzungskriterien}
	\begin{itemize}
		\item
	\end{itemize}
	
	
	\section{\Large PRODUKTEINSATZ}
	\subsection{Anwendungsbereiche}
	\begin{itemize}
		\item
	\end{itemize}
	\subsection{Zielgruppen}
	\begin{itemize}
		\item Die Zielgruppe sind Leute, 
		\begin{itemize}
			\item
		\end{itemize}
	\end{itemize}
	\subsection{Betriebsbedingungen}
	\begin{itemize}
		\item
	\end{itemize}
	
	
	\section{\Large PRODUKTÜBERSICHT}
	Gibt eine Übersicht über das Produkt, z.B. über alle wichtigen Geschäftsprozesse in Form eines Übersichtsdiagramms.
	\subsection{Usecase Diagramm}
		
		
	\section{\Large PRODUKTFUNKTIONEN}
	\subsection{Usecase-Beschreibungen}
	\begin{table}[H]
		\begin{tabular}{|p{8cm}|p{8cm}|}
			\hline
			\textbf{GEO-01 } \\ 
			\hline
			\textbf{ID :}\centering & UC01  \\ \hline 
			\textbf{Title :}\centering & Test \\ \hline 
			\textbf{Description :}\centering & Test \\ \hline 
			\textbf{Trigger :}\centering & Test \\ \hline 
			\textbf{Primary Actor :} \centering & User \\ \hline 
			\textbf{Preconditions :}\centering & 
			1. Test \newline 
			2. Test \newline
			3. Test	\\ \hline 
			\textbf{Postconditions :}\centering &  
			1. Test \newline 
			2. Test \\ \hline
			\textbf{Other Use Cases :}\centering & - \\ \hline  
			\textbf{Main Success Scenario :}\centering & 
			1. Test \newline
			3. Test \\ \hline  
			\textbf{Extensions :}\centering & - \\ \hline  
			\textbf{Priority :}\centering & High \\ \hline  
		\end{tabular}
	\end{table}	

	\subsection{Aktivitätsdiagramm}
		
	\subsection{Sequenzdiagramm}
	
	
\newpage
	\section{\Large PRODUKTDATEN}

	\subsection{Analyseklassendiagramm}	

	\subsection{Paketdiagramm}

	\subsection{Domänenklassendiagramm}

	\section{\Large PRODUKTLEISTUNGEN}

	\section{\Large QUALITÄTSANFORDERUNGEN}

	
	\section{\Large BENUTZEROBERFLÄCHE}
	Es gibt eine Rolle und das ist die des Users der das Prgoramm ausführt (GUI).

	\section{\Large NICHTFUNKTIONALE ANFORDERUNGEN}
	Es werden alle Anforderungen aufgeführt, die sich nicht auf die Funktionalität, \textbf{ die Leistung} und \textbf{ die Benutzungsoberfläche} beziehen, z.B. :
	\begin{itemize}
		\item
	\end{itemize} 

	
	\section{\Large TECHNISCHE PRODUKTUMGEBUNG}
   	In diesem Kapitel wird die technische Umgebung des Produkts beschrieben.\\
   	Bei Client / Server-Anwendungen ist die Umgebung jeweils für Clients und Server getrennt anzugeben.

	\subsection{Software}
	\begin{itemize}
		\item
	\end{itemize}

	\subsection{Hardware}
	\begin{itemize}
		\item
	\end{itemize}

	\subsection{Orgware}
	\begin{itemize}
		\item
	\end{itemize}	

	\subsection{Produkt-Schnittstellen}
	\begin{itemize}
		\item
	\end{itemize}
	
	\section{\Large SPEZIELLE ANFORDERUNGEN AN DIE ENTWICKLUNGS-UMGEBUNG}
	Entwicklung- und Testumgebung des Frontends: Siehe 10 Technische Produktentwicklung 
	\subsection{Software}
	\subsection{Hardware}
	\subsection{Orgware}
	\subsection{Entwicklungsschnittstellen}
	
	
	\section{\Large GLIEDERUNG IN TEILPRODUKTE}
	
	
	\section{\Large ERGÄNZUNGEN}

	
\newpage
	\section{\Large GLOSSAR}
	In diesem Kapitel wird die spezifische Sprache des Auftraggebers wie \textbf{ Kürzel } und \textbf{ Fachbegriffe } beschrieben, z.B. :
	\begin{itemize}
		\item
	\end{itemize}

		